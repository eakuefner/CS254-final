\section{Memory models and Caching}
\note{Give a brief introduction describing the problem here. In garbage collection section we deal with the data cache and memory. The section subsection deals with instruction caches.}
\subsection{Garbage Collection}
Garbage collection involves reclaiming the precious heap space available after an object is no longer needed by a managed runtime system. This has been an active research agrea and most of the effecient implementations involve use technique called compaction. Compaction moves around the live objects in the heap into a compact region so that there is less fragmentation of the heap. This is a costly operation and usually introduces noticable pause times during program execution. In \cite{wegiel08} Wegiel et al describe a way of using virtual memory system to perform this operation with minimum overhead. The main idea here is to exploit the fact that the dead objects often appear in clusters in the heap. Using virtual memory mapping/unmapping API provided by the operating system, the virtual page which in which all the objects are guaranteed to be dead is unmapped from the heap. Though this technique has its limitations since the collection phase doesn't always free all the dead objects, it is shown to be an efficient technique in collecting the objects from tenured region of the heap in a generational garbage collector. \note{Not complete}

\subsection{Instruction caches}
In \cite{lin11} Lin et al describes a methodology of arrangement of language runtime code in memory to enable greater performance in cache-sensitive architectures. \note{expand this even more}
