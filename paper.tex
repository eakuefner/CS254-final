%-----------------------------------------------------------------------------
%
%               Template for sigplanconf LaTeX Class
%
% Name:         sigplanconf-template.tex
%
% Purpose:      A template for sigplanconf.cls, which is a LaTeX 2e class
%               file for SIGPLAN conference proceedings.
%
% Guide:        Refer to "Author's Guide to the ACM SIGPLAN Class,"
%               sigplanconf-guide.pdf
%
% Author:       Paul C. Anagnostopoulos
%               Windfall Software
%               978 371-2316
%               paul@windfall.com
%
% Created:      15 February 2005
%
%-----------------------------------------------------------------------------


\documentclass[preprint,nocopyrightspace,10pt]{sigplanconf}

% The following \documentclass options may be useful:
%
% 10pt          To set in 10-point type instead of 9-point.
% 11pt          To set in 11-point type instead of 9-point.
% authoryear    To obtain author/year citation style instead of numeric.

\usepackage{amsmath}
\usepackage{graphicx}
\usepackage{xspace}
\usepackage{color}

\newcommand{\ignore}[1]{}
\newcommand{\note}[1]{{\color{red}{\textbf{[#1]}}}}
\newcommand{\vsp}{\vspace{.1in}}
\newcommand{\nvsp}{\vspace{-.1in}}


\usepackage{amsmath}
\usepackage{color}
\usepackage{hyperref}
\hypersetup{colorlinks=true,citecolor=cyan}

\begin{document}

%\conferenceinfo{WXYZ '05}{date, City.} 
%\copyrightyear{2005} 
%\copyrightdata{[to be supplied]} 

%\titlebanner{banner above paper title}        % These are ignored unless
%\preprintfooter{short description of paper}   % 'preprint' option specified.

\title{Examining the Connection Between Hardware Architecture and Virtual Machine Design}
%\subtitle{Subtitle Text, if any}

%\authorinfo{Name1}
%           {Affiliation1}
%           {Email1}
\authorinfo{Ethan A. Kuefner\and Madhukar N. Kedlaya}
           {University of California, Santa Barbara}
           {\{eakuefner,mkedlaya\}@cs.ucsb.edu}

\maketitle

%TODO: Remove papers from this list as they are cited.
\nocite{lin11,wegiel08,sullivan03,jo12}

\begin{abstract}
Virtual machines for programming language interpretation have become popular despite the long-standing generality
that compiled code is faster than interpreted code. By viewing interpreters as virtual machines, we can exploit
the similarity between virtual machines and the underlying hardware architectures on which they run to take
advantage of optimizations implicit in hardware and improve the performance of virtual machines. In this paper,
branch prediction and spatial locality are reviewed as two critical optimizations to be exploited by VMs.
\end{abstract}

%\category{CR-number}{subcategory}{third-level}

%\terms
%term1, term2

%\keywords
%keyword1, keyword2

\section{Introduction}

Interpreters have played a central role in the study of programming languages for as long as programming
languages have been studied. Explicit in McCarthy's definition of LISP \cite{mccarthy60} is the notion of an
evaluator, a program designed to run (evaluate) other programs. For many years, interpreters have been used
to study the design and semantics of programming languages. Until recently, however, languages like FORTRAN,
COBOL, and later C remained popular for programming, because of how fast programs compiled to machine code
from these languages were.

Java, which appeared in 1995, was one of the first interpreted languages to use a so-called 
\emph{virtual machine}, and has proved popular. 
Java programs are compiled down to Java bytecode, which is machine code for the Java Virtual Machine.
Since Java, bytecode interpreters have become commonplace. Indeed, even in the original academic circles where
interpreters were conceived and made popular, researchers have connected the old notions of
interpretation as evaluation with the modern idea of interpreter as virtual machine \cite{ager03}. With
the advent of the virtual machine language runtime an opportunity has arisen to rectify the main flaw of
interpretation alluded to above: the speed gap between interpretation of high-level code and execution of
machine code. One of the major ways that this has been done is through exploitation of properties of the
underlying hardware architectures on which these interpreters run.

We divide this paper into three main parts. In the first part, we discuss three papers related to improving
hardware branch prediction accuracy for virtual machines, and in the second part, we discuss two papers
related to exploiting memory models and caching. Lastly, we conclude by discussing the possibility of furthering
this work in several interesting directions.

\section{Branch Prediction}
Programming language design has always been a tricky area of research. More importantly the design of the runtime plays an important role in enabling interesting language features while maintaining acceptable performance. When we consider the implementation of modern language runtimes, implementing a quick interpreter has mostly been the first choice. The main reason for this is the ease of implementation and the ease of extensibility. Unfortunately, interpreters come with a large performance overhead. Classical big-step style interpreters which traverse through the AST of a program suffer from this. An improvement over this design are the bytecode based interpreters where the bytecodes are fetched, decoded and then executed in a loop. The classical way of impementing this is using switch statement matching the opcode with various possible cases and executing the corresponding opcode handler. 

The major drawback of this approach is the bytecode fetching and branching overhead. Other alternatives to the switch-case approach like direct threading, indirect threading also suffer from the similar branch mis-prediction overhead. There branches that are taken depend on the flow of execution or the stream of bytecodes of the program being executed. It does not follow any pattern which is very much the foundation of branch prediction strategies implemented in the hardware. It is important to note here that the branches that we are describing are the ones in the interpreter  loop and should not be confused with the branches in the program being interpreted. The problem with branch mis-prediction is even more exagerated in the case of Dynamic Scripting Languages (DSLs) where the opcode handlers for a dynamic instruction have to first determine the types of the opcodes and them perform operations based on the observed types. This is generally implemented using a \emph{if-else} chain or a \emph{switch-case} construct adding to the number of branches mis-predicted during execution.
\subsection{Dynamo and DynamoRIO}
Dynamo is an pseudo interpreter which interprets machine code and specializes the hot traces that are executed at runtime. It follows the same philosophy as a JIT compiler but at a much lower level. Sullivan et. al presented DynamoRIO, based on IA-32 version of Dynamo, as a solution to reduce the interpretation overhead. Using Dynamo or DynamoRIO naively as a binary optimizer for any interpreter does not yield any speedup. This is because DynamoRIO relies on its trace collection heuristic to optimize a binary and as mentioned above the trace of execution of any interpreter depends on the program being interpreted which is unpredictable at runtime. To solve this problem DynamioRIO infrastucture provides APIs to language runtime developers to instrument their interpreters with hooks to a special traceing framework. The hooks provide signals to the underlying framework when to start and stop the trace collection. The idea here is to make sure that the trace that is collected matches the program that is being interpreted rather than the interpreter that is interpreting it. This gives lot more information to the tracing infrastructure to work on and optimize. 
\subsection{Context Threading}
\subsection{Instruction replication and Superinstructions}
\section{Memory models and Caching}
\subsection{Garbage Collection}
\subsection{Instruction caches}
% We recommend abbrvnat bibliography style.
\pagebreak
\bibliographystyle{abbrvnat}

% The bibliography should be embedded for final submission.

\bibliography{paper.bib}

\appendix
\section{Sample Homework Problem}
\href{https://bitbucket.org/eakuefner/cs162interpreter}{Here} is a working bytecode compiler and interpreter toolchain for a
simple programming language, written in C++. Use this DynamoRIO \href{http://www.dynamorio.org/tutorial.html}{tutorial} to
instrument the interpreter with hooks for straight-line code as described in this paper, and then compare the performance
with and without the tracer enabled. What is the speedup that you observe?

\section{More Ideas}
In this appendix we list a number of additional ideas along these lines that we might be interested in exploring. You know how
to reach us!
\begin{itemize}
\item What does it mean to design instruction caches for functional programs? Are there additional invariants of programs in
languages like Haskell that we might exploit?
\item Above we discussed hinting hardware branch predictors---could the same reasoning be applied to caching? In other words,
would it be possible to design hardware to which a programmer could indicate that they plan on later using a chunk of memory,
or using a chunk of memory particularly frequently?
\item Would it be possible to design hardware features that remove some of the complexities that make adaptive/realtime program
analyses difficult?
\item Could we design a system for context-sensitive memory access reordering? In other words, could we apply something like in
\cite{jo12} to hardware memory access reordering?
\end{itemize}
\end{document}
